\documentclass{article}
\usepackage[utf8]{inputenc}
\usepackage[english]{babel}
 

\usepackage{amsfonts}

\usepackage{blindtext}
\usepackage{amssymb}


\usepackage{amsthm}

%\newtheorem{theorem}{Theorem}
\newtheorem{theorem}{Theorem}[section]
\newtheorem{corollary}{Corollary}[theorem]
\newtheorem{lemma}[theorem]{Lemma}

\theoremstyle{remark}
\newtheorem*{remark}{Remark}

\theoremstyle{definition}
\newtheorem{definition}{Definition}[section]

\renewcommand\qedsymbol{$\blacksquare$}
\newcommand{\simplies}{\DOTSB\Longrightarrow}
\newcommand{\lsimplies}{\DOTSB\Longleftarrow}

\title{General Topology Notes}
\author{Shimal Harichurn}
\date{January 2017}


\begin{document}

\maketitle

\newpage

\tableofcontents

\newpage

\addcontentsline{toc}{section}{Preface}
\section*{Preface}

\begin{flushleft}

The purpose of these sets of notes are to act as a completion of references, from which I can look back on at any time. They are meant to integrate with other references, such as the textbooks I've used or the conversations I've had with others, and are not stand-alone. 
\end{flushleft}

\begin{flushleft}
One of the motivations for setting up these notes, was that I found it difficult and cumbersome to keep track of all the results and proofs I'd made and it was hindering my progress, as a result I had proved in one theorem, would be needed at some later point, and flipping back and forth through pages in my book was frustrating, hence the need to move to a more streamlined method of note-taking
\end{flushleft}

\begin{flushleft}
Furthermore most of the results I put here, will hopefully be unique and contributed by myself. I will only make comments or remarks on certain theorems in here that can be found elsewhere, but I will not copy/paste any theorems in here (that is something I strive not to do)

\end{flushleft}

\begin{flushleft}
The emphasis here is to provide a unique set of results, theorems and views on General Topology, as I progress through learning it, and to act as a supplementary to the usual reference. Only definitions will be copy and pasted here for completeness. Any theorems and results that occur in \textit{Munkres} (in the main text and not the exercises), shall be left for the reader to look up in \textit{Munkres}. Remember that this is a supplementary text to \textit{Munkres}, which together will hopefully form one complete reference.

\end{flushleft}

\begin{flushleft}
\textit{Reference Text: Topology : A First Course - by Munkres}

\end{flushleft}

\newpage

\section{Definition of a Topology}
\medskip
\begin{definition}
Given any set $X$, a topology $\mathcal{T}$ on $X$, is a collection of subsets of $X$, such that

\begin{itemize}
  \item $\emptyset \in \mathcal{T}$ and $X \in \mathcal{T}$
  \item \textit{Arbitrary} unions of subcollections of $\mathcal{T}$ are contained in $\mathcal{T}$. In other words if $\mathcal{K} \subset \mathcal{T}$, then $$\left( \ \bigcup_{U \in \mathcal{K}}U \ \right) \in \mathcal{T}$$
  \item The intersection of the elements of any \textit{finite} subcollection, $\mathcal{J} \subset \mathcal{T}$, is in $\mathcal{T}$. i.e. $$\left( \ \bigcap_{U \in \mathcal{J}}U \ \right) \in \mathcal{T}$$
\end{itemize}
\end{definition}

\medskip
\hrulefill
\begin{flushleft}
A few remarks on the definition of a topology. 
\end{flushleft}
\\ \\
\textbf{\underline{Question:}} 
\ If $\mathcal{T}$ is a topology on a set $X$, then why in the definition of a topology, do we have the condition that only the intersection of finite sub-collections of $\mathcal{T}$ be in $\mathcal{T}$? As opposed to arbitrary subcollections?
\\ \\
\textbf{\underline{Answer:}} 
\ Let's assume for the moment that we can take arbitrary subcollections of the topology $\mathcal{T}$. And let's look at an example where $X = \mathbb{R}$ and $\mathcal{T}$ is the standard topology on $\mathbb{R}$ (i.e we are looking at the topological space $(\mathbb{R}, \mathcal{T})$.
\\ \\
Let $\mathcal{K}$ be the subcollection of $\mathcal{T}$ of all sets of the form $(\frac{-1}{n}, \frac{1}{n})$. In other words $$\mathcal{K} = \left\{\left(\frac{-1}{n}, \frac{1}{n}\right) \ \middle| \ \frac{-1}{n}, \frac{1}{n} \in \mathbb{R}\right\}$$.

Clearly $\mathcal{K} \subset \mathcal{T}$, and $\mathcal{K}$ is not finite, but $$\bigcap_{U \in \mathcal{K}} U = \{0\} \not\in \mathcal{T}$$

\newpage


\section{Basis of a Topology}

\subsection{Definitions (In progress)}

\begin{definition}[Basis]

\end{definition}

\begin{definition}[Subbasis]

\end{definition}

\begin{definition}[Topology generated by a basis]

\end{definition}

\newpage


\subsection{Conceptual Understanding}

The following provides a basic conceptual framework that was not so apparent to me initially. 

\medskip

Given $\mathcal{B}$ as a basis for a topology on a set $X$, the following facts hold. 

\begin{itemize}
  \item Every basis $\mathcal{B}$, generates one and only one unique topology $\mathcal{T}$
  \item A topology $\mathcal{T}$ on a set $X$ can have many bases. In other worlds a topology $\mathcal{T}$ on a set $X$ can have a collection of bases $\{\mathcal{B_{\alpha}}\}$
  \item Any set $X$ can have multiple bases.
  \item Any set $X$ can have multiple topologies
\end{itemize}

\hrulefill

\subsection{Basic Theorems}

\medskip

\begin{theorem}[Uniqueness Theorem]
If two topologies $\mathcal{T}$ and $\mathcal{T'}$ have the same basis $\mathcal{B}$, then $\mathcal{T} = \mathcal{T'}$
\end{theorem}

\begin{proof}
TBA. I will put the proof here as soon as possible.
\end{proof}

\medskip

\begin{theorem}
If $\mathcal{B}$ is a basis for a topology $\mathcal{T}$ on a set $X$, every element of $\mathcal{B}$ is an element of $\mathcal{T}$
\end{theorem}

\begin{proof}
Take $B \in \mathcal{B}$. By definition of a basis, we have $B \subset X$. Now let's take a look at the definition for a topology $\mathcal{T}$ generated by a basis $\mathcal{B}$. $$\mathcal{T} = \{ U \subset X \ | \ \forall x \in U, \ \exists B \in \mathcal{B} \ \text{s.t.} \ x \in B \subset U\}$$
\\ 
If we take $U = B$, we can see that each $B \in \mathcal{B}$ satisfies the requirement for $U$ to be an element of $\mathcal{T}$ vacuously and thus each $B \in \mathcal{B} \simplies B \in \mathcal{T}$, and hence every element of $\mathcal{B}$ is an element of $\mathcal{T}$.
\end{proof}



\begin{corollary}
If $\mathcal{B}$ is a basis for a topology $\mathcal{T}$ on a set $X$, then $\mathcal{B} \subset \mathcal{T}$
\end{corollary}

\newpage

\begin{theorem}
If $X$ is any set and $\mathcal{T}$ is a topology on $X$. $\mathcal{T}$ is a basis for itself.
\end{theorem}

\begin{proof}
Since $\mathcal{T}$ is a topology on $X$, we have $X \in \mathcal{T}$, thus it satisfies property $(1)$ in the definition of a basis. Now since $\mathcal{T}$ is a topology on $X$, we know that the intersection of any finite subcollection of $\mathcal{T}$ is in $\mathcal{T}$.
\\ \\
So let's suppose $B_1 \in \mathcal{T}$ and $B_2 \in \mathcal{T}$, by what we've recalled above, $B_1 \cap B_2 \in \mathcal{T}$. Put $\gamma = B_1 \cap B_2$. Clearly $\gamma \subset B_1 \cap B_2$, and thus we've satisfied property $(2)$ in the definition of a basis. 
\end{proof}




\newpage

\subsection{Translating between bases and topologies (In Progress)}

- Lemma 13.1 in Munkres
- Page next to it

\newpage

\subsection{Comparing topologies}

\bigskip

\begin{theorem}
Let $X$ be any set and $\mathcal{T}_1, \mathcal{T}_2$ topologies on $X$ with bases $\mathcal{B}_1$ and $\mathcal{B}_2$ respectively. If $\mathcal{B}_1 \subset \mathcal{B}_2$, then $\mathcal{T}_1 \subset \mathcal{T}_2$
\end{theorem}

\begin{proof}
$\mathcal{T}_1$ equals the collection of all unions of elements of $\mathcal{B}_1$, and $\mathcal{T}_2$ equals the collection of all unions of elements of $\mathcal{B}_2$. Therefore we have
 $$\mathcal{T}_1 = \left\{U \ \middle| \ U = \bigcup_{B \in \mathcal{K}}B \ \text{ where } \ \mathcal{K} \subset \mathcal{B}_1\right\}$$

$$\mathcal{T}_2 = \left\{U \ \middle| \ U = \bigcup_{B \in \mathcal{K}}B \ \text{ where } \ \mathcal{K} \subset \mathcal{B}_2\right\}$$
\\
Since $\mathcal{B}_1 \subset \mathcal{B}_2$, every possible $\mathcal{K} \subset \mathcal{B}_1 \simplies \mathcal{K} \subset \mathcal{B}_2$, which thus implies $B \in \mathcal{K} \subset \mathcal{B}_1 \simplies B \in\mathcal{K} \subset \mathcal{B}_2$, so that $U = \bigcup_{B \in \mathcal{K}}B \in \mathcal{T}_1 \simplies U \in \mathcal{T}_2$ and thus we have $\mathcal{T}_1 \subset \mathcal{T}_2$.
\end{proof}

\smallskip \begin{flushleft}
\textbf{\underline{Question:}} \ Does the converse $\mathcal{T}_1 \subset \mathcal{T}_2 \simplies \mathcal{B}_1 \subset \mathcal{B}_2$ hold?
\end{flushleft}
\\ \\
\textbf{\underline{Disproof:}} \ (Answer from math.stackexchange)\ \\ \\ $\mathcal{T}_2$ is a basis in itself so obviously this does not hold.
\\ \\
Explicit counter example: Let $X = \mathbb R$, $\mathcal{T}_1$ the Euclidean topology, $\mathcal{T}_2$ the discrete topology. Let $\mathcal{B}_1$ be the open intervals and $\mathcal B_2$ the singletons.
\medskip
\begin{remark}
Ah okay, I also realized $\mathcal{T}_1 \subset \mathcal{T}_2 \simplies \mathcal{B}_1 \subset \mathcal{B}_2$ holds for some bases, but \textbf{not for all} though. If we take $\mathcal{T}_1 = \mathcal{B}_1$ (which can can because every topology is a basis for itself) and $\mathcal{T}_2 = \mathcal{B}_2$ we can see it holds trivially. I think the confusion I faced was in forgetting that each topology can have multiple bases. 
\end{remark}

\hrulefill

\bigskip
\medskip

\begin{theorem}
Let $X$ be any set and $\mathcal{T}_1, \mathcal{T}_2$ topologies on $X$ with bases $\mathcal{B}_1$ and $\mathcal{B}_2$ respectively. If $\mathcal{B}_1 \subset \mathcal{T}_2$, then $\mathcal{T}_1 \subset \mathcal{T}_2$
\end{theorem}

\begin{proof}
$\mathcal{T}_1$ equals the collection of all unions of elements of $\mathcal{B}_1$. Since $\mathcal{B}_1 \subset \mathcal{T}_2$ (by hypotheses), the union of all elements of $\mathcal{B}_1$ is contained in $\mathcal{T}_2$ (since $\mathcal{T}_2$ is a topology), thus $\mathcal{T}_1$ is contained in $\mathcal{T}_2$. 
\\ \\
Expressed more rigorously $$\mathcal{T}_1 = \left\{U \ \middle| \ U = \bigcup_{B \in \mathcal{K}}B \ \text{ where } \ \mathcal{K} \subset \mathcal{B}_1\right\}$$
\\
So since  $\mathcal{B}_1 \subset \mathcal{T}_2$, $\mathcal{B}_1$ is a subcollection $\mathcal{T}_2$. Take $\mathcal{K} \subset \mathcal{B}_1 \subset \mathcal{T}_2$, then $\bigcup_{B \in \mathcal{K}}B \in \mathcal{T}_2$ and thus $U \in \mathcal{T}_1 \simplies U \in \mathcal{T}_2$ and we have $\mathcal{T}_1 \subset \mathcal{T}_2$.
\end{proof}

\newpage

\section{Order, Subspace and Product Topologies (In Progress)}

\bigskip

\subsection{The Order Topology (In Progress)}
\bigskip
\subsection{The Subspace Topology}
\begin{definition}[Subspace Topology]
Let $(X, \mathcal{T})$ be a topological space. If $Y \subset X$, then the collection $$\mathcal{T}_Y = \{ Y \cap U \ | \  U \in \mathcal{T}\}$$ is a topology on $Y$, called the \textbf{subspace topology}.
\end{definition}
\medskip
\begin{remark}
Note that the subspace topology $\mathcal{T}_Y$ in the definition above is completely dependant on the topology $\mathcal{T}$ on $X$. For example if we take $X = \mathbb{R}$, then if $\mathcal{T}$ is the standard topology on $\mathbb{R}$ and $\mathcal{T'}$ is the $K$-topology on $\mathbb{R}$, then the subspace topology $\mathcal{T}_Y$ with respect to the standard topology on $\mathbb{R}$ would be different to the subspace topology $\mathcal{T'}_Y$ with respect to the $K$-topology on $\mathbb{R}$, as the standard topology and $K$-topology have different bases and thus different elements. \\ \\Though one subspace topology could be a subset of another if the corresponding topology on $X$ is a subset of the other, as we see in the theorem below.
\end{remark}
\medskip
\begin{theorem}
If $\mathcal{T}$ and $\mathcal{T'}$ are topologies on $X$ and $\mathcal{T} \subset \mathcal{T'}$ where $\mathcal{T} \neq \mathcal{T'}$ and $Y \subset X$, then we have $\mathcal{T}_Y \subset \mathcal{T'}_Y$
\end{theorem}

\begin{proof}
\textit{(Munkres Exercise 16.2)} We have $\mathcal{T} \subset \mathcal{T'}$. So given $Y \subset X$, we have the following subspace topologies $$\mathcal{T}_Y = \{Y \cap U \ | \ U \in \mathcal{T}\} \ \ \text{and} \ \ \mathcal{T'}_Y = \{ Y \cap U \ | \ U \in \mathcal{T'}\}$$
Now since $U \in \mathcal{T} \simplies U \in \mathcal{T'}$, we have $Y \cap U \in \mathcal{T} \simplies Y \cap U \in \mathcal{T'}$, and thus we have $\mathcal{T}_Y \subset \mathcal{T'}_Y$.
\end{proof}
\medskip
\begin{theorem}
If $(X, \mathcal{T})$ is a topological space and $Y \subset X$ with the subspace $(Y, \mathcal{T}_Y)$, such that $Y \in \mathcal{T}$, then $ \mathcal{T}_Y \subset \mathcal{T}$
\end{theorem}

\begin{proof}
The proof for this is trivial since, $Y \in \mathcal{T}$, and $\mathcal{T}$ is a topology on $X$ $Y \cap U \in \mathcal{T}$ holds for any $U \in \mathcal{T}$ so that $\mathcal{T}_Y = \{ Y \cap U \ | U \in \mathcal{T}\} \subset \mathcal{T}$ since every $Y \cap U \in  \mathcal{T}_Y \simplies Y \cap U \in  \mathcal{T}$.
\end{proof}

\smallskip \begin{flushleft}
\textbf{\underline{Question:}} \ If $(X, \mathcal{T})$ is a topological space and $Y \subset X$ and with $(Y, \mathcal{T}_Y)$ being a subspace. Is $\mathcal{T}_Y \subset \mathcal{T}$?
\end{flushleft}
\\ \\
\textbf{\underline{Disproof:}} We construct an explicit counterexample. Take $X = \mathbb{R}$, and let $\mathcal{T}$, be the standard topology on $\mathbb{R}$, and let $Y = \{0\} \subset \mathbb{R}$. Then $\mathcal{T}_Y = \left\{ \{0\} \cap U \ | \ U \in \mathcal{T} \right\}$. Now take $(-1, 1) \in \mathcal{T}$, then $\{0\} \cap (-1, 1) = \{0\} \in \mathcal{T}_Y$. But $\{0\} \not\in \mathcal{T}$. So that $U \in \mathcal{T}_Y \not\simplies U \in \mathcal{T}$. Therefore $\mathcal{T}_Y \not\subset \mathcal{T}.$


\bigskip
\subsection{The Product Topology}

\begin{definition}
Let $(X, \mathcal{T}_X)$ and $(Y, \mathcal{T}_Y)$ be topological spaces. The \textbf{product topology} on $X \times Y$ is the topology having as basis the collection $$\mathcal{B} = \left\{ \ U \times V \ \middle| \ U \in \mathcal{T}_X \ \ \text{and} \ \ V \in \mathcal{T}_Y\right\}$$
\end{definition}

\begin{remark}
The product topology, $\mathcal{T}$, is thus $$\mathcal{T} = \left\{ \ \bigcup_{U \in \mathcal{T}_X, \ V \in \mathcal{T}_Y} U \times V \right\}
\end{remark}
\newpage

\section{Closed Sets, Limit Points and Boundaries}

\subsection{Definitions}

\begin{definition}[Closed Set]
A subset $A \subset X$ of a topological space $(X , \mathcal{T})$ is said to be \textbf{closed} if $X-A \in \mathcal{T}$ (i.e if its complement is open) 
\end{definition}

\begin{remark}
Now the fact that a subset $A \subset X$ of a topological space $(X, \mathcal{T})$, can be either open, closed, open and closed, or neither open nor closed is what confuses some. But the confusion, I believe, can be cleared if we expressed mathematically what these terms mean.

\begin{enumerate}
  \item \textbf{(Open)} $A$ is \textit{open} if $A \in \mathcal{T}$ 
  \item \textbf{(Closed)} $A$ is \textit{closed} if $X - A \in \mathcal{T}$
  \item \textbf{(Open and Closed)} $A$ is \textit{both open and closed} if $A \in \mathcal{T} \  \text{and} \ X - A \in \mathcal{T}$
  \item \textbf{(Not open and  not closed)} $A$ is \textit{neither open nor closed} if $A \not\in \mathcal{T} \  \text{and} \ \\X - A \not\in \mathcal{T}$ 
\end{enumerate}
\end{remark}

\medskip

\begin{definition}[Closure]
Given a subset $A \subset X$ of a topological space $(X , \mathcal{T})$, the \textbf{closure} of $A$, denoted by $Cl(A)$ or $\bar{A}$ is defined to be the intersection of all closed sets containing $A$.

$$Cl(A) = \bar{A} = \bigcap_{i , \  A \subset W_i} W_i \ \ \ \text{ where } \ \ \ X - W_i \in \mathcal{T}$$

\end{definition}

\begin{definition}[Interior]
Given a subset $A \subset X$ of a topological space $(X , \mathcal{T})$, the \textbf{interior} of $A$, denoted by $Int(A)$ is defined to be the union of all open sets contained in $A$.

$$Int(A) = \bigcup_{U \in \mathcal{T} , \  U \subset A} U $$

\end{definition}

\medskip

\begin{remark}[Neighbourhoods]
Most mathematicians shorten the statement \textit{"$U$ is an open set containing $x$"} to the following statement \textbf{"$U$ is a neighbourhood of $x$"}.
\\ \\
In other words if we have $x \in X$, where $(X, \mathcal{T})$ is some topological space, then 
\begin{center}
$\text{"U \ is \ a \ neighbourhood \ of \ x"} \iff x \in U \in \mathcal{T}$ 
\end{center}
\end{remark}

\medskip

\newpage

\begin{definition}[Limit Point]
Given a subset $A \subset X$ of a topological space $(X , \mathcal{T})$, and if $x \in X$, we say that $x$ is a \textbf{limit point} of $A$ if $x$ belongs to the closure of $A - \{x\}$. In other words $x \in Cl\left(A - \{x\}\right) = \overline{A - \{x\}}$.
\\ \\
Equivalently, we say that $x$ is a \textbf{limit point} of $A$, if every \textit{neighbourhood} of $x$ (i.e. open set $U \in \mathcal{T}$ where $x \in U$) intersects $A$ in some point other than $x$ itself. \\
So if $\{U_{\alpha}\}$ is the collection of all neighbourhoods of $x$, $x$ is a limit point of $A$ if for all $\alpha$ we have: $$U_{\alpha} \cap A = V \ \ \ \text{where} \ \ \ x \in V \  \text{and} \  V \neq \{x\}$$
\end{definition}

\medskip

\begin{definition}[Boundary]
Given a subset $A \subset X$ of a topological space $(X , \mathcal{T})$, we define the \textbf{boundary} of $A$ by $$Bd \ (A) = \overline{A} \ \cap \ \overline{X-A}$$
\end{definition}

\newpage

\subsection{Theorems}

\begin{theorem}
The set $S$ is closed if and only if $Cl(S) = S$
\end{theorem}

\begin{proof}[Proof]
$(\simplies)$. Suppose $S$ is closed. Let $\{V_{\alpha}\}$ be the collection of all closed sets containing $S$. In other words $S \subset V_{\alpha}$ for every $\alpha$. Now $S = V_{\alpha}$ for some $\alpha$ since $S$ is closed and clearly $S \subset S$, so that we have $Cl(S) = \bigcap_{\alpha}V_{\alpha} = S$
\\ \\ 
$(\lsimplies)$. Conversely suppose $Cl(S) = S$ and let $\{V_{\alpha}\}$ be the collection of all closed sets containing $S$. By definition $Cl(S) = \bigcap_{\alpha}V_{\alpha}$ and since $S = Cl(S)$, we have $S = \bigcap_{\alpha}V_{\alpha}$. And since arbitrary intersections of closed sets are closed, it follows that $S$ is closed.
\end{proof}
\medskip
\begin{theorem}
If $A$ is closed in $Y$ and $Y$ is closed in $X$, then $A$ is closed in $X$ (Ex 17.2 in Munkres)
\end{theorem}

\begin{proof}
We make use of the following lemma in this proof. 
\textit{Lemma (Theorem 17.2 in Munkres)} "Let $Y$ be a subspace of $X$. Then a set $A$ is closed in $Y$ if and only if it equals the intersection of a closed set of $X$ with $Y$".
\\ \\
So $A = Y \cap B$, where $B$ is a closed set of $X$. Since $A \subset Y$, it follows that $B = A$, and thus $A$ is closed in $X$.
\end{proof}
\medskip
\begin{theorem}
If $A$ is closed in $X$ and $B$ is closed in $Y$, then $A \times B$ is closed in $X \times Y$ (Ex 17.3 in Munkres)
\end{theorem}

\begin{proof}
We'll let $\mathcal{T}_X$ and $\mathcal{T}_Y$ denote the topologies on $X$ and $Y$ respectively. Since $A$ is closed in $X$, we have $X - A \in \mathcal{T}_X$ and since $B$ is closed in $Y$ we have $Y - B \in \mathcal{T}_Y$ Now by definition of the product topology $\mathcal{T}$ on $X \times Y$, $$\mathcal{T} = \left\{ \ \bigcup_{U \in \mathcal{T}_X, \ V \in \mathcal{T}_Y} U \times V \right\}$$ and by elementary set operations $$(X\times Y)-(A\times B)=\Big((X- A)\times Y\Big)\cup\Big(X\times(Y - B)\Big)$$.
Now since $X - A \in \mathcal{T}_X$ and by the definition of a topology, $Y \in \mathcal{T}_Y$, we have $\left((X- A)\times Y\right) \in \mathcal{T}$ and since $Y - B \in \mathcal{T}_Y$ and by the definition of a topology, $X \in \mathcal{T}_X$, we have $\left(X\times(Y - B)\right) \in \mathcal{T}$. Hence, since $\mathcal{T}$, the product topology, is a topology by definition we have$$\Big((X- A)\times Y\Big)\cup\Big(X\times(Y - B)\Big) = (X\times Y)-(A\times B) \in \mathcal{T}$$ so that $A \times B$ is closed in $X \times Y$.
\end{proof}

\newpage

\section{Haursdoff Spaces}
\bigskip
\begin{definition}[Haursdoff Spaces]
\\ \\
A topological space $(X, \mathcal{T})$ is called a \textbf{Haursdoff Space} if for each distinct pair $x_1, x_2 \in X$, there exists $U_1$ and $U_2$ in $\mathcal{T}$ such that $x_1 \in U_1$ and $x_2 \in U_2$ and $U_1 \cap U_2 = \emptyset$
\end{definition}
\medskip
\begin{remark}
The motivation for the definition and classification of topological spaces as \textit{Haursdoff Spaces}, is because in arbitrary topological spaces, one-point sets may not be closed and sequences can converge to more than one point. 
\\ \\
Haursdoff came up with the condition for Haursdoff Spaces to ensure that one-point sets are closed and sequences converge only to one point in Haursdoff Spaces
\end{remark}

\begin{definition}[$T_1$ Axiom]
The condition that finite point sets be closed is called the \textbf{$T_1$ Axiom}

\end{definition}

\end{document}



