\documentclass{article}
\usepackage[utf8]{inputenc}
\usepackage[english]{babel}
 

\usepackage{amsfonts}
\usepackage{amsmath}
\usepackage{blindtext}
\usepackage{amssymb}


\usepackage{amsthm}

%\newtheorem{theorem}{Theorem}
\newtheorem{theorem}{Theorem}[section]
\newtheorem{corollary}{Corollary}[theorem]
\newtheorem{lemma}[theorem]{Lemma}

\theoremstyle{remark}
\newtheorem*{remark}{Remark}

\theoremstyle{definition}
\newtheorem{definition}{Definition}[section]

\renewcommand\qedsymbol{$\blacksquare$}
\newcommand{\simplies}{\DOTSB\Longrightarrow}
\newcommand{\lsimplies}{\DOTSB\Longleftarrow}

\title{General Topology Notes}
\author{Shimal Harichurn}
\date{January 2017}


\begin{document}

\maketitle

\newpage

\tableofcontents

\newpage

\addcontentsline{toc}{section}{Preface}
\section*{Preface}

\begin{flushleft}

The purpose of these sets of notes are to act as a completion of references, from which I can look back on at any time. They are meant to integrate with other references, such as the textbooks I've used or the conversations I've had with others, and are not stand-alone. 
\end{flushleft}

\begin{flushleft}
One of the motivations for setting up these notes, was that I found it difficult and cumbersome to keep track of all the results and proofs I'd made and it was hindering my progress, as a result I had proved in one theorem, would be needed at some later point, and flipping back and forth through pages in my book was frustrating, hence the need to move to a more streamlined method of note-taking
\end{flushleft}

\begin{flushleft}
Furthermore most of the results I put here, will hopefully be unique and contributed by myself. I will only make comments or remarks on certain theorems in here that can be found elsewhere, but I will not copy/paste any theorems in here (that is something I strive not to do)

\end{flushleft}

\begin{flushleft}
The emphasis here is to provide a unique set of results, theorems and views on General Topology, as I progress through learning it, and to act as a supplementary to the usual reference. Only definitions will be copy and pasted here for completeness. Any theorems and results that occur in \textit{Munkres} (in the main text and not the exercises), shall be left for the reader to look up in \textit{Munkres}. Remember that this is a supplementary text to \textit{Munkres}, which together will hopefully form one complete reference. 

\end{flushleft}

\begin{flushleft}
Every proof contained in these set of notes is mine, unless otherwise specified.

\end{flushleft}


\begin{flushleft}
\textit{Reference Text: Topology : A First Course - by Munkres}

\end{flushleft}

\newpage

\section{Definition of a Topology}
\bigskip
\begin{definition}[\textbf{Topology}]
Given any set $X$, a topology $\mathcal{T}$ on $X$, is a collection of subsets of $X$, such that

\begin{itemize}
  \item $\emptyset \in \mathcal{T}$ and $X \in \mathcal{T}$
  \item \textit{Arbitrary} unions of subcollections of $\mathcal{T}$ are contained in $\mathcal{T}$. In other words if $\mathcal{K} \subset \mathcal{T}$, then $$\left( \ \bigcup_{U \in \mathcal{K}}U \ \right) \in \mathcal{T}$$
  \item The intersection of the elements of any \textit{finite} subcollection, $\mathcal{J} \subset \mathcal{T}$, is in $\mathcal{T}$. i.e. $$\left( \ \bigcap_{U \in \mathcal{J}}U \ \right) \in \mathcal{T}$$
\end{itemize}
\end{definition}

\bigskip
\hrulefill
\medskip
\begin{flushleft}
A few remarks on the definition of a topology. 
\end{flushleft}
\\ \\
\textbf{\underline{Question:}} 
\ If $\mathcal{T}$ is a topology on a set $X$, then why in the definition of a topology, do we have the condition that only the intersection of finite sub-collections of $\mathcal{T}$ be in $\mathcal{T}$? As opposed to arbitrary subcollections?
\\ \\
\textbf{\underline{Answer:}}  
\ (By Ted Shifrin) Let's assume for the moment that we can take arbitrary subcollections of the topology $\mathcal{T}$. And let's look at an example where $X = \mathbb{R}$ and $\mathcal{T}$ is the standard topology on $\mathbb{R}$ (i.e we are looking at the topological space $(\mathbb{R}, \mathcal{T})$.
\\ \\
Let $\mathcal{K}$ be the subcollection of $\mathcal{T}$ of all sets of the form $(\frac{-1}{n}, \frac{1}{n})$. In other words $$\mathcal{K} = \left\{\left(\frac{-1}{n}, \frac{1}{n}\right) \ \middle| \ \frac{-1}{n}, \frac{1}{n} \in \mathbb{R}\right\}$$.

Clearly $\mathcal{K} \subset \mathcal{T}$, and $\mathcal{K}$ is not finite, but $$\bigcap_{U \in \mathcal{K}} U = \{0\} \not\in \mathcal{T}$$

\newpage


\section{Basis and Subbasis for a Topology}

\bigskip

\subsection{Definitions}

\begin{definition}[\textbf{Basis}]
If $X$ is a set, a \textbf{basis}, for a topology on $X$ is, a collection $\mathcal{B}$ of subsets of $X$ (called \textit{basis elements)} such that 
\\ \\ 
(1) $\forall x \in X$, $\exists$ at least one $B \in \mathcal{B}$ such that $x \in B$ \\
(2) If $x \in B_1 \cap B_2$, then $\exists B_3$ such that $x \in B_3$ and $B_3 \subset B_1 \cap B_2$
\end{definition}

\bigskip

\begin{definition}[\textbf{Topology generated by a basis}]
We define the \textit{topology $\mathcal{T}$ generated by a basis $\mathcal{B}$} on a set $X$ as follows

$$\mathcal{T} = \left\{ U \subset X \ \middle| \ \forall x \in U, \ \exists B \in \mathcal{B} \ \ \text{such  that} \ \ x \in B \subset U\ \right\}$$
\end{definition}

\medskip

\begin{remark}[\textbf{Important!}]
Alternatively, and more easily, we can define the \textbf{topology generated by a basis} to be the collection of all unions of basis elements

$$\mathcal{T} = \left\{ \bigcup_{B \in \mathcal{K}} B \ \middle| \ \mathcal{K} \subset \mathcal{B} \right\}$$
\end{remark}
\medskip
\begin{remark}
This alternative definition above is actually a theorem in \textit{Munkres}, however it the one I use all the time, as it is a vast simplification of the original definition.
\end{remark}

\bigskip

\begin{definition}[\textbf{Subbasis}]
A \textbf{subbasis}, $\mathcal{S}$, for a topology on $X$, is a collection of subsets of $X$, whose union equals $X$. 

$$\mathcal{S} = \left\{S_{\alpha} \subset X \ \middle| \ \bigcup_{\alpha} S_{\alpha} = X \right\}


\end{definition}

\bigskip

\begin{definition}[\textbf{Basis generated by a subbasis}]
The basis $\mathcal{B}$ generated by a subbasis $\mathcal{S}$ consists of all finite intersections of elements of $\mathcal{S}$
$$\ \mathcal{B} = \left\{\bigcap_{\alpha \in J} S_{\alpha} \ \middle| \ S_{\alpha} \in \mathcal{S} \ \text{and $J$ is a finite indexing set} \ \right\}$$
\end{definition}

\bigskip

\begin{definition}[\textbf{Topology generated by a subbasis}]
The \textit{topology $\mathcal{T}$ generated by a subbasis $\mathcal{S}$} on a set $X$ is defined to be the collection of all unions of finite intersections of elements of $\mathcal{S}$

$$\mathcal{T} = \left\{ \ \bigcup_{\alpha \in J} \left( \ \bigcap_{i \in I} S_i \right) \  \middle| \  S_i \in \mathcal{S} \ \text{and $I$ is a finite indexing set and $J = \{ I \ | \  \text{$I$ is a finite indexing set}\}$} \right\}$$
\end{definition}



\newpage


\subsection{Conceptual Understanding}

The following provides a basic conceptual framework that was not so apparent to me initially. 

\medskip

Given $\mathcal{B}$ as a basis for a topology on a set $X$, the following facts hold. 

\begin{itemize}
  \item Every basis $\mathcal{B}$, generates one and only one unique topology $\mathcal{T}$
  \item A topology $\mathcal{T}$ on a set $X$ can have many bases. In other worlds a topology $\mathcal{T}$ on a set $X$ can have a collection of bases $\{\mathcal{B_{\alpha}}\}$
  \item Any set $X$ can have multiple bases.
  \item Any set $X$ can have multiple topologies
\end{itemize}

\hrulefill

\subsection{Basic Theorems}

\medskip

\begin{theorem}[Uniqueness Theorem]
If two topologies $\mathcal{T}$ and $\mathcal{T'}$ have the same basis $\mathcal{B}$, then $\mathcal{T} = \mathcal{T'}$
\end{theorem}

\begin{proof}
TBA. I will put the proof here as soon as possible.
\end{proof}

\medskip

\begin{theorem}
If $\mathcal{B}$ is a basis for a topology $\mathcal{T}$ on a set $X$, every element of $\mathcal{B}$ is an element of $\mathcal{T}$
\end{theorem}

\begin{proof}
Take $B \in \mathcal{B}$. By definition of a basis, we have $B \subset X$. Now let's take a look at the definition for a topology $\mathcal{T}$ generated by a basis $\mathcal{B}$. $$\mathcal{T} = \{ U \subset X \ | \ \forall x \in U, \ \exists B \in \mathcal{B} \ \text{s.t.} \ x \in B \subset U\}$$
\\ 
If we take $U = B$, we can see that each $B \in \mathcal{B}$ satisfies the requirement for $U$ to be an element of $\mathcal{T}$ vacuously and thus each $B \in \mathcal{B} \simplies B \in \mathcal{T}$, and hence every element of $\mathcal{B}$ is an element of $\mathcal{T}$.
\end{proof}



\begin{corollary}
If $\mathcal{B}$ is a basis for a topology $\mathcal{T}$ on a set $X$, then $\mathcal{B} \subset \mathcal{T}$
\end{corollary}

\newpage

\begin{theorem}
If $X$ is any set and $\mathcal{T}$ is a topology on $X$. $\mathcal{T}$ is a basis for itself.
\end{theorem}

\begin{proof}
Since $\mathcal{T}$ is a topology on $X$, we have $X \in \mathcal{T}$, thus it satisfies property $(1)$ in the definition of a basis. Now since $\mathcal{T}$ is a topology on $X$, we know that the intersection of any finite subcollection of $\mathcal{T}$ is in $\mathcal{T}$.
\\ \\
So let's suppose $B_1 \in \mathcal{T}$ and $B_2 \in \mathcal{T}$, by what we've recalled above, $B_1 \cap B_2 \in \mathcal{T}$. Put $\gamma = B_1 \cap B_2$. Clearly $\gamma \subset B_1 \cap B_2$, and thus we've satisfied property $(2)$ in the definition of a basis. 
\end{proof}




\newpage

\subsection{Translating between bases and topologies (In Progress)}

- Lemma 13.1 in Munkres
- Page next to it

\newpage

\subsection{Comparing topologies}

\bigskip

\begin{theorem}
Let $X$ be any set and $\mathcal{T}_1, \mathcal{T}_2$ topologies on $X$ with bases $\mathcal{B}_1$ and $\mathcal{B}_2$ respectively. If $\mathcal{B}_1 \subset \mathcal{B}_2$, then $\mathcal{T}_1 \subset \mathcal{T}_2$
\end{theorem}

\begin{proof}
$\mathcal{T}_1$ equals the collection of all unions of elements of $\mathcal{B}_1$, and $\mathcal{T}_2$ equals the collection of all unions of elements of $\mathcal{B}_2$. Therefore we have
 $$\mathcal{T}_1 = \left\{U \ \middle| \ U = \bigcup_{B \in \mathcal{K}}B \ \text{ where } \ \mathcal{K} \subset \mathcal{B}_1\right\}$$

$$\mathcal{T}_2 = \left\{U \ \middle| \ U = \bigcup_{B \in \mathcal{K}}B \ \text{ where } \ \mathcal{K} \subset \mathcal{B}_2\right\}$$
\\
Since $\mathcal{B}_1 \subset \mathcal{B}_2$, every possible $\mathcal{K} \subset \mathcal{B}_1 \simplies \mathcal{K} \subset \mathcal{B}_2$, which thus implies $B \in \mathcal{K} \subset \mathcal{B}_1 \simplies B \in\mathcal{K} \subset \mathcal{B}_2$, so that $U = \bigcup_{B \in \mathcal{K}}B \in \mathcal{T}_1 \simplies U \in \mathcal{T}_2$ and thus we have $\mathcal{T}_1 \subset \mathcal{T}_2$.
\end{proof}

\smallskip \begin{flushleft}
\textbf{\underline{Question:}} \ Does the converse $\mathcal{T}_1 \subset \mathcal{T}_2 \simplies \mathcal{B}_1 \subset \mathcal{B}_2$ hold?
\end{flushleft}
\\ \\
\textbf{\underline{Disproof:}} \ (Answer from math.stackexchange)\ \\ \\ $\mathcal{T}_2$ is a basis in itself so obviously this does not hold.
\\ \\
Explicit counter example: Let $X = \mathbb R$, $\mathcal{T}_1$ the Euclidean topology, $\mathcal{T}_2$ the discrete topology. Let $\mathcal{B}_1$ be the open intervals and $\mathcal B_2$ the singletons.
\medskip
\begin{remark}
Ah okay, I also realized $\mathcal{T}_1 \subset \mathcal{T}_2 \simplies \mathcal{B}_1 \subset \mathcal{B}_2$ holds for some bases, but \textbf{not for all} though. If we take $\mathcal{T}_1 = \mathcal{B}_1$ (which can can because every topology is a basis for itself) and $\mathcal{T}_2 = \mathcal{B}_2$ we can see it holds trivially. I think the confusion I faced was in forgetting that each topology can have multiple bases. 
\end{remark}

\hrulefill

\bigskip
\medskip

\begin{theorem}
Let $X$ be any set and $\mathcal{T}_1, \mathcal{T}_2$ topologies on $X$ with bases $\mathcal{B}_1$ and $\mathcal{B}_2$ respectively. If $\mathcal{B}_1 \subset \mathcal{T}_2$, then $\mathcal{T}_1 \subset \mathcal{T}_2$
\end{theorem}

\begin{proof}
$\mathcal{T}_1$ equals the collection of all unions of elements of $\mathcal{B}_1$. Since $\mathcal{B}_1 \subset \mathcal{T}_2$ (by hypotheses), the union of all elements of $\mathcal{B}_1$ is contained in $\mathcal{T}_2$ (since $\mathcal{T}_2$ is a topology), thus $\mathcal{T}_1$ is contained in $\mathcal{T}_2$. 
\\ \\
Expressed more rigorously $$\mathcal{T}_1 = \left\{U \ \middle| \ U = \bigcup_{B \in \mathcal{K}}B \ \text{ where } \ \mathcal{K} \subset \mathcal{B}_1\right\}$$
\\
So since  $\mathcal{B}_1 \subset \mathcal{T}_2$, $\mathcal{B}_1$ is a subcollection $\mathcal{T}_2$. Take $\mathcal{K} \subset \mathcal{B}_1 \subset \mathcal{T}_2$, then $\bigcup_{B \in \mathcal{K}}B \in \mathcal{T}_2$ and thus $U \in \mathcal{T}_1 \simplies U \in \mathcal{T}_2$ and we have $\mathcal{T}_1 \subset \mathcal{T}_2$.
\end{proof}

\newpage

\section{Order and Subspace Topologies (In Progress)}

\bigskip

\subsection{The Order Topology (In Progress)}
\bigskip
\subsection{The Subspace Topology}
\begin{definition}[Subspace Topology]
Let $(X, \mathcal{T})$ be a topological space. If $Y \subset X$, then the collection $$\mathcal{T}_Y = \{ Y \cap U \ | \  U \in \mathcal{T}\}$$ is a topology on $Y$, called the \textbf{subspace topology}.
\end{definition}
\medskip
\begin{remark}
Note that the subspace topology $\mathcal{T}_Y$ in the definition above is completely dependant on the topology $\mathcal{T}$ on $X$. For example if we take $X = \mathbb{R}$, then if $\mathcal{T}$ is the standard topology on $\mathbb{R}$ and $\mathcal{T'}$ is the $K$-topology on $\mathbb{R}$, then the subspace topology $\mathcal{T}_Y$ with respect to the standard topology on $\mathbb{R}$ would be different to the subspace topology $\mathcal{T'}_Y$ with respect to the $K$-topology on $\mathbb{R}$, as the standard topology and $K$-topology have different bases and thus different elements. \\ \\Though one subspace topology could be a subset of another if the corresponding topology on $X$ is a subset of the other, as we see in the theorem below.
\end{remark}
\medskip
\begin{theorem}
If $\mathcal{T}$ and $\mathcal{T'}$ are topologies on $X$ and $\mathcal{T} \subset \mathcal{T'}$ where $\mathcal{T} \neq \mathcal{T'}$ and $Y \subset X$, then we have $\mathcal{T}_Y \subset \mathcal{T'}_Y$
\end{theorem}

\begin{proof}
\textit{(Munkres Exercise 16.2)} We have $\mathcal{T} \subset \mathcal{T'}$. So given $Y \subset X$, we have the following subspace topologies $$\mathcal{T}_Y = \{Y \cap U \ | \ U \in \mathcal{T}\} \ \ \text{and} \ \ \mathcal{T'}_Y = \{ Y \cap U \ | \ U \in \mathcal{T'}\}$$
Now since $U \in \mathcal{T} \simplies U \in \mathcal{T'}$, we have $Y \cap U \in \mathcal{T} \simplies Y \cap U \in \mathcal{T'}$, and thus we have $\mathcal{T}_Y \subset \mathcal{T'}_Y$.
\end{proof}
\medskip
\begin{theorem}
If $(X, \mathcal{T})$ is a topological space and $Y \subset X$ with the subspace $(Y, \mathcal{T}_Y)$, such that $Y \in \mathcal{T}$, then $ \mathcal{T}_Y \subset \mathcal{T}$
\end{theorem}

\begin{proof}
The proof for this is trivial since, $Y \in \mathcal{T}$, and $\mathcal{T}$ is a topology on $X$ $Y \cap U \in \mathcal{T}$ holds for any $U \in \mathcal{T}$ so that $\mathcal{T}_Y = \{ Y \cap U \ | U \in \mathcal{T}\} \subset \mathcal{T}$ since every $Y \cap U \in  \mathcal{T}_Y \simplies Y \cap U \in  \mathcal{T}$.
\end{proof}

\smallskip \begin{flushleft}
\textbf{\underline{Question:}} \ If $(X, \mathcal{T})$ is a topological space and $Y \subset X$ and with $(Y, \mathcal{T}_Y)$ being a subspace. Is $\mathcal{T}_Y \subset \mathcal{T}$?
\end{flushleft}
\\ \\
\textbf{\underline{Disproof:}} We construct an explicit counterexample. Take $X = \mathbb{R}$, and let $\mathcal{T}$, be the standard topology on $\mathbb{R}$, and let $Y = \{0\} \subset \mathbb{R}$. Then $\mathcal{T}_Y = \left\{ \{0\} \cap U \ | \ U \in \mathcal{T} \right\}$. Now take $(-1, 1) \in \mathcal{T}$, then $\{0\} \cap (-1, 1) = \{0\} \in \mathcal{T}_Y$. But $\{0\} \not\in \mathcal{T}$. So that $U \in \mathcal{T}_Y \not\simplies U \in \mathcal{T}$. Therefore $\mathcal{T}_Y \not\subset \mathcal{T}.$


\bigskip
\newpage
\section{The Product and Box Topologies}

\bigskip

\subsection{The (Finite) Product Topology}

\bigskip

\begin{definition}[\textbf{Finite Product Topology}]
Let $(X, \mathcal{T}_X)$ and $(Y, \mathcal{T}_Y)$ be topological spaces. The \textbf{product topology} on $X \times Y$ is the topology having as basis the collection $$\mathcal{B} = \left\{ \ U \times V \ \middle| \ U \in \mathcal{T}_X \ \ \text{and} \ \ V \in \mathcal{T}_Y\right\}$$
\end{definition}

\begin{remark}
The product topology, $\mathcal{T}$, is thus $$\mathcal{T} = \left\{ \ \bigcup_{U \in \mathcal{T}_X, \ V \in \mathcal{T}_Y} U \times V \right\}$$
\end{remark}

\bigskip


\newpage

\subsection{The Product Topology}

\medskip
Note that before reading this, it is best to take a look at definition of an arbitrary indexing set, and arbitrary Cartesian Products with respect to this arbitrary indexing set $J$ (which will be used throughout the definitions below), in \textit{Munkres} on pages $113$ and $114$.

\bigskip
\begin{definition}[\textbf{Projection Mapping}]
Let $$\pi_{\beta} : \prod_{\alpha \in J} X_{\alpha} \to X_{\beta}$$ be the function assigning to each element of the product spaces its $\beta$th coordinate. $$\pi_{\beta}((x_{\alpha})_{\alpha \in J}) = x_{\beta}$$ This is called the projection mapping 
\end{definition}
\bigskip
\begin{definition}[\textbf{Product Topology}]
Let $\mathcal{S}_{\beta}$ denote the collection $$\mathcal{S}_{\beta} = \left\{ \pi_{\beta}^{-1}(U_{\beta}) \ | \ U_{\beta} \text{ is open in} \ X_{\beta}\right\}$$ and let $\mathcal{S}$ denote the union of these collections, $$\mathcal{S} = \bigcup_{\beta \in J}S_{\beta}$$ The topology generated by the subbasis $\mathcal{S}$ is called the \textbf{product topology}. In this topology $\prod_{\alpha \in J}X_{\alpha}$ is called a \textbf{product space}.
\end{definition}



\bigskip

\begin{remark}[\textbf{1}]
Note that if $\ U_{\beta} \text{ is open in} \ X_{\beta}\right\}$, then $$\pi_{\beta}^{-1}(U_{\beta}) =  U_{\beta} \times \prod_{\alpha \in I} X_{\alpha}$$ where $I = J - \{\beta\}$ (the indexing set, without the index $\beta$. Also note that $U_{\beta} \times \prod_{\alpha \in I} X_{\alpha}$, is open in $\prod_{\alpha \in J} X_{\alpha} $
\end{remark}
\medskip
\begin{remark}[\textbf{2}]
In the definition of the subbasis for the product topology above, in the collection $\mathcal{S}_{\beta}$ we are taking all possible open sets $U_{\beta}$ from every $X_{\beta}$. Thus $\mathcal{S}_{\beta}$ is a collection formed by taking all possible open sets $U_{\beta} \subset X_{\beta}$ ($\beta$ fixed).
\end{remark}

\newpage


\begin{remark}[\textbf{3}]
In simple English terms $\mathcal{S}$ equals the union of all possible inverse projection mappings of open sets in every $X_{\beta}$, so $\mathcal{S}$ equals all possible inverse projection mappings of open sets from all the topological spaces $\{X_{\beta}\}_{\beta \in J}$
\end{remark}

\medskip

\begin{remark}[\textbf{4}]
The reason we leave the $\beta$ index on $\pi^{-1}_{\beta}$ is to act as a reminder as to which $X_{\beta}$ the projection mapping is mapping to. 

Also the $U_{\beta}$ is redundant as it is just there to indicate that $U_{\beta}$ is open with respect to $X_{\beta}$, we could just as easily write $\mathcal{S}_{\beta}$ as $\mathcal{S}_{\beta} \left\{ \pi_{\beta}^{-1}(U) \ | \ U \text{ is open in} \ X_{\beta}\right\}$ 
\end{remark}

\newpage

\subsection{The Box Topology}

\bigskip

\begin{definition}[\textbf{The Box Topology}]
Let $\{X_{\alpha}\}_{\alpha \in J}$ be an indexed family of topological spaces. Let us take as a basis for a topology on the product space $$\prod_{\alpha \in J}X_{\alpha}$$ the collection of all sets of the form $$\prod_{\alpha \in J}U_{\alpha}$$ where $U_{\alpha}$ is open in $X_{\alpha}$, for each $\alpha \in J$. The topology generated by this basis is called the \textbf{box topology}.
\end{definition}

\begin{remark}
Expressed formally the basis $\mathcal{B}$ for the box topology is the following
$$\mathcal{B} = \left\{ \ \prod_{\alpha \in J}U_{\alpha} \ \middle| \ \text{$U_{\alpha}$ is open in $X_{\alpha}$} \ \right\}$$
And the box topology generated by the basis $\mathcal{B}$ above is
$$\mathcal{T}_B = \left\{ \ \bigcup_{B \in \mathcal{K}} B \ \middle| \ B = \prod_{\alpha \in J}U_{\alpha} \ \text{and} \ \mathcal{K} \subset \mathcal{B} \right\}$$
\end{remark}

\newpage

\begin{theorem}
Suppose the topology on each space $X_{\alpha}$ is given by a basis $\mathcal{B}_{\alpha}$. The collection of all sets of the form $$\prod_{\alpha \in J} B_{\alpha}$$ where $B_{\alpha} \in \mathcal{B}_{\alpha}$ for each ${\alpha}$, will serve as a basis for the box topology on $\prod_{\alpha \in J} X_{\alpha}$

\end{theorem}

\begin{proof}
The topological space we are dealing with here is $\left(\prod_{\alpha \in J} X_{\alpha}, \mathcal{T}_B\right)$, where $\mathcal{T}_B$ is the box topology on  $\prod_{\alpha \in J} X_{\alpha}$. Let $\mathcal{B}_{\text{box}}$ denote the standard basis for the box topology.
\\ \\
Put $$\mathcal{F} = \left\{ \ \prod_{\alpha \in H} B_{\alpha} \ \middle| \ B_{\alpha} \in \mathcal{B}_{\alpha} \ \text{for each $\alpha$} \ \right\}.$$ Then take $U \in \mathcal{T}_B$, and take $x \in U$. Since $U = \bigcup_{ B \in \mathcal{K}}B$ where $B = \prod_{\alpha \in J}U_{\alpha}$ and $U_{\alpha}$ is open in $X_{\alpha}$ and $\mathcal{K} \subset \mathcal{B}_{\text{box}}$ we then have $x \in B$ for some $B \in \mathcal{K}$. In other words $x \in \prod_{\alpha \in J}U_{\alpha}$ for some indexing set $J$.
\\ \\
Now since each $U_{\alpha}$ is open in $X_{\alpha}$, we have $U_{\alpha} = \bigcup_{\gamma \in I}B_{\gamma}$ where $B_{\gamma}$ are basis elements of $X_{\alpha}$. Thus $$x \in \prod_{\alpha \in J}\left(\bigcup_{\gamma \in I}B_{\gamma}\right)_{\alpha}$$
\\ \\
By elementary set theory we have $$\left(\bigcup_{i \in I} V_i\right) \times \left(\bigcup_{j \in J} W_j\right) = \bigcup_{\langle i, j \rangle I \times J}\left(V_i \times W_j\right)$$
\\
So that $$\prod_{\alpha \in J}\left(\bigcup_{\gamma}B_{\gamma}\right)_{\alpha} \  = \  \bigcup_{\langle i_1, i_2, ... \rangle \in I_1 \times I_2 \times ...} \underbrace{\left(B_{i_1} \times B_{i_2} \times  ...\right}_{|J| \ \text{times}})$$.
\\ 
Choose $H = I_1 \times I_2 \times I_3 \times ... \ \ \ $\\ Then $\exists \prod_{\alpha \in H}B_{\alpha} \in \mathcal{F}$ such that $x \in \prod_{\alpha \in H}B_{\alpha}$. It also follows that $$\prod_{\alpha \in J}\left(\bigcup_{\gamma}B_{\gamma}\right)_{\alpha} \subset U$$ Since $\prod_{\alpha \in J}\left(\bigcup_{\gamma}B_{\gamma}\right)_{\alpha} = \prod_{\alpha \in J}U_{\alpha} = B \subset U$. Thus by \textit{Lemma 13.2} in Munkres it follows that $\mathcal{F}$ is a basis for the box topology on $\prod_{\alpha \in J} X_{\alpha}$.
\end{proof}

\newpage



\section{Closed Sets, Limit Points and Boundaries}
\bigskip
\subsection{Definitions}

\bigskip

\begin{definition}[\textbf{Closed Set}]
A subset $A \subset X$ of a topological space $(X , \mathcal{T})$ is said to be \textbf{closed} if $X-A \in \mathcal{T}$ (i.e if its complement is open) 
\end{definition}

\begin{remark}
Now the fact that a subset $A \subset X$ of a topological space $(X, \mathcal{T})$, can be either open, closed, open and closed, or neither open nor closed is what confuses some. But the confusion, I believe, can be cleared if we expressed mathematically what these terms mean.

\begin{enumerate}
  \item \textbf{(Open)} $A$ is \textit{open} if $A \in \mathcal{T}$ 
  \item \textbf{(Closed)} $A$ is \textit{closed} if $X - A \in \mathcal{T}$
  \item \textbf{(Open and Closed)} $A$ is \textit{both open and closed} if $A \in \mathcal{T} \  \text{and} \ X - A \in \mathcal{T}$
  \item \textbf{(Not open and  not closed)} $A$ is \textit{neither open nor closed} if $A \not\in \mathcal{T} \  \text{and} \ \\X - A \not\in \mathcal{T}$ 
\end{enumerate}
\end{remark}

\medskip

\begin{definition}[\textbf{Closure}]
Given a subset $A \subset X$ of a topological space $(X , \mathcal{T})$, the \textbf{closure} of $A$, denoted by $Cl(A)$ or $\bar{A}$ is defined to be the intersection of all closed sets containing $A$.

$$Cl(A) = \bar{A} = \bigcap_{i , \  A \subset W_i} W_i \ \ \ \text{ where } \ \ \ X - W_i \in \mathcal{T}$$

\end{definition}

\begin{definition}[\textbf{Interior}]
Given a subset $A \subset X$ of a topological space $(X , \mathcal{T})$, the \textbf{interior} of $A$, denoted by $Int(A)$ is defined to be the union of all open sets contained in $A$.

$$Int(A) = \bigcup_{U \in \mathcal{T} , \  U \subset A} U $$

\end{definition}

\medskip

\begin{remark}[\textbf{Neighbourhoods}]
Most mathematicians shorten the statement \textit{"$U$ is an open set containing $x$"} to the following statement \textbf{"$U$ is a neighbourhood of $x$"}.
\\ \\
In other words if we have $x \in X$, where $(X, \mathcal{T})$ is some topological space, then 
\begin{center}
$\text{"U \ is \ a \ neighbourhood \ of \ x"} \iff x \in U \in \mathcal{T}$ 
\end{center}
\end{remark}

\medskip

\newpage

\begin{definition}[\textbf{Limit Point}]
Given a subset $A \subset X$ of a topological space $(X , \mathcal{T})$, and if $x \in X$,  we say that $x$ is a \textbf{limit point} of $A$, if every \textit{neighbourhood} of $x$ (i.e. open set $U \in \mathcal{T}$ where $x \in U$) intersects $A$ in some point other than $x$ itself. \\ \\ So if $\{U_{\alpha}\}$ is the collection of all neighbourhoods of $x$, $x$ is a limit point of $A$ if for all $\alpha$ we have: $$U_{\alpha} \cap A = V \ \ \ \text{where} \ \ \   V \neq \{x\}$$
\\ \\
Equivalently, we say that $x$ is a \textbf{limit point} of $A$ if $x$ belongs to the closure of $A - \{x\}$. In other words $x \in Cl\left(A - \{x\}\right) = \overline{A - \{x\}}$. \\

\end{definition}

\begin{remark}[Example]
Take $A =(0,1) \subset \mathbb{R}$ with the usual topology on $\mathbb{R}$. Then take $U$ to be a neighbourhood of $0$. Fix $\gamma > 0$ and $\epsilon > 0$, then every $U$ is of the form $(0 - \gamma, 0 + \epsilon)$, and thus for every $U$ we have, $U \cap A \neq \emptyset$, and $U \cap A \neq \{0\}$.
\end{remark}

\bigskip

\begin{definition}[\textbf{Boundary}]
Given a subset $A \subset X$ of a topological space $(X , \mathcal{T})$, we define the \textbf{boundary} of $A$ by $$Bd \ (A) = \overline{A} \ \cap \ \overline{X-A}$$
\end{definition}

\newpage

\subsection{Theorems}

\begin{theorem}
The set $S$ is closed if and only if $Cl(S) = S$
\end{theorem}

\begin{proof}[Proof]
$(\simplies)$. Suppose $S$ is closed. Let $\{V_{\alpha}\}$ be the collection of all closed sets containing $S$. In other words $S \subset V_{\alpha}$ for every $\alpha$. Now $S = V_{\alpha}$ for some $\alpha$ since $S$ is closed and clearly $S \subset S$, so that we have $Cl(S) = \bigcap_{\alpha}V_{\alpha} = S$
\\ \\ 
$(\lsimplies)$. Conversely suppose $Cl(S) = S$ and let $\{V_{\alpha}\}$ be the collection of all closed sets containing $S$. By definition $Cl(S) = \bigcap_{\alpha}V_{\alpha}$ and since $S = Cl(S)$, we have $S = \bigcap_{\alpha}V_{\alpha}$. And since arbitrary intersections of closed sets are closed, it follows that $S$ is closed.
\end{proof}
\medskip
\begin{theorem}
If $A$ is closed in $Y$ and $Y$ is closed in $X$, then $A$ is closed in $X$ (Ex 17.2 in Munkres)
\end{theorem}

\begin{proof}
We make use of the following lemma in this proof. 
\textit{Lemma (Theorem 17.2 in Munkres)} "Let $Y$ be a subspace of $X$. Then a set $A$ is closed in $Y$ if and only if it equals the intersection of a closed set of $X$ with $Y$".
\\ \\
So $A = Y \cap B$, where $B$ is a closed set of $X$. Since $A \subset Y$, it follows that $B = A$, and thus $A$ is closed in $X$.
\end{proof}
\medskip
\begin{theorem}
If $A$ is closed in $X$ and $B$ is closed in $Y$, then $A \times B$ is closed in $X \times Y$ (Ex 17.3 in Munkres)
\end{theorem}

\begin{proof}
We'll let $\mathcal{T}_X$ and $\mathcal{T}_Y$ denote the topologies on $X$ and $Y$ respectively. Since $A$ is closed in $X$, we have $X - A \in \mathcal{T}_X$ and since $B$ is closed in $Y$ we have $Y - B \in \mathcal{T}_Y$ Now by definition of the product topology $\mathcal{T}$ on $X \times Y$, $$\mathcal{T} = \left\{ \ \bigcup_{U \in \mathcal{T}_X, \ V \in \mathcal{T}_Y} U \times V \right\}$$ and by elementary set operations $$(X\times Y)-(A\times B)=\Big((X- A)\times Y\Big)\cup\Big(X\times(Y - B)\Big)$$
Now since $X - A \in \mathcal{T}_X$ and by the definition of a topology, $Y \in \mathcal{T}_Y$, we have $\left((X- A)\times Y\right) \in \mathcal{T}$ and since $Y - B \in \mathcal{T}_Y$ and by the definition of a topology, $X \in \mathcal{T}_X$, we have $\left(X\times(Y - B)\right) \in \mathcal{T}$. Hence, since $\mathcal{T}$, the product topology, is a topology by definition we have$$\Big((X- A)\times Y\Big)\cup\Big(X\times(Y - B)\Big) = (X\times Y)-(A\times B) \in \mathcal{T}$$ so that $A \times B$ is closed in $X \times Y$.
\end{proof}

\newpage

\begin{theorem}
If $U$ is open in $X$ and $A$ is closed in $X$, then $U-A$ is open in $X$, and $A-U$ is closed in $X$
\end{theorem}

\begin{proof}
Let $\mathcal{T}$ denote the topology on $X$. Then $U \in \mathcal{T}$ and $X-A \in \mathcal{T}$. Since $\mathcal{T}$ is a topology, we have $(X-A) \cap U \in \mathcal{T}$. Therefore $(X-A) \cap U$ is open so that $X - ((X-A) \cap U)$ is closed. \\ \\ By the algebra of sets, $X - ((X-A) \cap U) = X - (X-A) \cap X-U = X \cap A \cap (X-U) = A \cap (X-U) = X \cap (A-U) = A- U$. Therefore $A-U$ is closed. 
\\ \\
And $(X-A) \cap U = (X \cap U) - A = U-A$. Therefore $U-A$ is open and hence $U-A \in \mathcal{T}$
\end{proof}

\begin{remark}
\textbf{This theorem is quite important}, and essentially says the following. 
\begin{itemize}
  \item Any open set minus a closed set is open.
  \item Any closed set minus an open set is closed.
\end{itemize}
\end{remark}

\bigskip

\begin{remark}[Comment]
I proved this theorem in John Dory's on 06 January 2017, while eating butternut and spinach and having a cappuccino
\end{remark}
\bigskip
\begin{theorem}
If $A$ and $B$ are subsets of a topological space $(X, \mathcal{T})$ and $A \subset B$, then $\overline{A} \subset \overline{B}$ (Ex 17.6 in Munkres)
\end{theorem}

\begin{proof}
We'll use the notation $U_x$ to be an open set of $X$ containing $x$, (i.e $U_x := x \in U$ and $U \in \mathcal{T})$. Now $x \in \overline{A}$ if for every $U_x \in \mathcal{T}$ we have $U_x \cap A \neq \emptyset$. And since $A \subset B$, it follows that if $U_x \cap A \neq \emptyset$, then $U_x \cap B \neq \emptyset$, so every open set containing $x$ intersecting $A$, also intersects $B$, hence $x \in \overline{A} \simplies x \in \overline{B}$, and we have $\overline{A} \subset \overline{B}.$
\end{proof}

\bigskip

\begin{theorem}
If $A$ and $B$ are subsets of a topological space $(X, \mathcal{T})$, then $\overline{A \cup B} = \overline{A} \cup \overline{B}$
\end{theorem}

\begin{proof}
We'll again use the notation, $U_x$ to be an open set of $X$ containing $x$, (i.e $U_x := x \in U$ and $U \in \mathcal{T})$. \\ \\Now $x \in \overline{A \cup B}$ if every $U_x \cap (A \cup B) \neq \emptyset \iff (U_x \cap A) \cup (U_x \cap B) \neq \emptyset$. (The logical equivalence used here is by simple set operations).
\\ \\
And we have $x \in \overline{A}$ if $U_x \cap A \neq \emptyset$ for every $U_x \in \mathcal{T}$, and similarly  $x \in \overline{B}$ if $U_x \cap B \neq \emptyset$ for every $U_x \in \mathcal{T}$, so that $x \in \overline{A} \cup \overline{B}$ if $(U_x \cap A \neq \emptyset) \lor (U_x \cap B \neq \emptyset).$
\\ \\
Logically we have $(U_x \cap A \neq \emptyset) \lor (U_x \cap B \neq \emptyset) \iff (U_x \cap A) \cup (U_x \cap B) \neq \emptyset$, and hence it follows by the biconditional  above that $x \in \overline{A \cup B} \simplies  x \in \overline{A} \cup \overline{B}$ and that $x \in \overline{A} \cup \overline{B} \simplies x \in \overline{A \cup B}$ and thus we have $\overline{A \cup B} = \overline{A} \cup \overline{B}$.
\end{proof}

\newpage

\begin{theorem}
Take $A \subset X$ and let $(X, \mathcal{T}_X)$ be a topological space. Take $x \in \bar{A}$, and let $W$ be a neighbourhood of $x$. Then $A \cap W$ is nonempty
\end{theorem}

\begin{proof}
If $x \in A$, then we are done. However if $x \not\in A$, then $x$ is a limit point of $A$, then by \textit{Theorem 17.5} in Munkres, $x \in \bar{A}$ if and only if every open set $U$ containing $x$ intersects $A$, thus it follows that $A \cap W$ must be nonempty.
\end{proof}

\newpage

\section{Haursdoff Spaces (In Progress)}
\bigskip
\begin{definition}[\textbf{Haursdoff Spaces}]
\\ \\
A topological space $(X, \mathcal{T})$ is called a \textbf{Haursdoff Space} if for each distinct pair $x_1, x_2 \in X$, there exists $U_1$ and $U_2$ in $\mathcal{T}$ such that $x_1 \in U_1$ and $x_2 \in U_2$ and $U_1 \cap U_2 = \emptyset$
\end{definition}
\medskip
\begin{remark}
The motivation for the definition and classification of topological spaces as \textit{Haursdoff Spaces}, is because in arbitrary topological spaces, one-point sets may not be closed and sequences can converge to more than one point. 
\\ \\
Haursdoff came up with the condition for Haursdoff Spaces to ensure that one-point sets are closed and sequences converge only to one point in Haursdoff Spaces
\end{remark}

\begin{definition}[$T_1$ Axiom]
The condition that finite point sets be closed is called the \textbf{$T_1$ Axiom}

\end{definition}

\begin{theorem}
Every order topology is Haursdoff
\end{theorem}

\begin{proof}
In progress
\end{proof}

\begin{theorem}
A subspace of a Haursdoff space is Haursdoff.
\end{theorem}

\begin{proof}
Let $(X, \mathcal{T}_X)$ be a Haursdoff space, and put $Y \subset X$ so that $(Y, \mathcal{T}_Y)$ is a subspace. Now take $x_1, x_2 \in Y$ such that $x_1 \neq x_2$. Since $(X, \mathcal{T}_X)$ is Haursdoff space, $\exists U_1, U_2 \in \mathcal{T}_X$ such that $x_1 \in U_1$, and $x_2 \in U_2$ and $U_1 \cap U_2 = \emptyset$. Then we have $Y \cap U_1 \in \mathcal{T}_Y$ and $Y \cap U_2 \in \mathcal{T}_Y$, with $x_1 \in Y \cap U_1$ and $x_2 \in Y \cap U_21$. Finally $( Y \cap U_1) \cap (Y \cap U_2) = Y \cap (U_1 \cap U_2) = Y \cap \emptyset = \emptyset$. And thus $(Y, \mathcal{T}_Y)$ is Haursdoff.
\end{proof}

\medskip

\begin{theorem}
The product of two Haursdoff Spaces is Haursdoff.
\end{theorem}

\begin{proof}
Let $(X, \mathcal{T}_X)$ and $(Y, \mathcal{T}_Y)$ be two Haursdoff spaces. We need to prove $(X \times Y, \mathcal{T})$, where $\mathcal{T}$ is the product topology on $X \times Y$ is Haursdoff. \\ \\ By definition we have $$\mathcal{T} = \left\{\bigcup_{U \in \mathcal{T}_X, \ V \in \mathcal{T}_Y} U \times V \right\}$$ Now take $(x_1, y_1) \in X \times Y$ and $(x_2, y_2) \in X \times Y$ such that $(x_1, y_1) \neq (x_2, y_2)$. Thus we have either $x_1 \neq x_2$ or $y_1 \neq y_2$ (or both). Assume without loss of generality that $x_1 \neq x_2$. Since $(X, \mathcal{T}_X)$ is Haursdoff, $\exists \ U_1, U_2$ such that $x_1 \in U_1$ and $x_2 \in U_2$ and $U_1 \cap U_2 = \emptyset$. And take $V_1, V_2 \in \mathcal{T}_Y$ not necessarily disjoint such that $y_1 \in V_1$ and $y_2 \in V_2$. Then $(x_1, y_1) \in U_1 \times V_1$ and $(x_2, y_2) \in U_2 \times V_2$. And we have $(U_1 \times V_1) \cap  (U_2 \times V_2) = (U_1 \cap U_2) \times (V_1 \cap V_2) = \emptyset \times (V_1 \cap V_2) = \emptyset$. The proof is similar if $y_1 \neq y_2$, and thus $(X \times Y, \mathcal{T})$, where $\mathcal{T}$ is Haursdoff.
\end{proof}

\newpage

\section{Continuous Functions}
\medskip
\begin{definition}[\textbf{Continuity}]
Let $(X, \mathcal{T}_X)$ and $(Y, \mathcal{T}_Y)$ be topological spaces. A function $f : X \to Y$ is said to be \textbf{continuous} if for each open subset $V$ of $Y$, the set $f^{-1}(V)$ is an open subset of $X$. \\ \\ In other words $f$ is continuous if for each $V \in \mathcal{T}_Y$, $f^{-1}(V) \in \mathcal{T}_X$
\end{definition}



\begin{remark}
Recall that $f^{-1}(V) = \{ x \in X \ | \  f(x) \in V \}$, and $f^{-1}(V) = \emptyset$ if $V \cap f(X) = \emptyset$
\end{remark}

\bigskip

\subsection{Proving continuity of functions}

\begin{enumerate}
  \item If the \textbf{topology} $\mathcal{T}_Y$ of the range space $Y$, \textbf{is given by a basis $\mathcal{B}$}, then to prove continuity of $f$ it suffices to show that the inverse image of every \textit{basis element} is open. \\ \\
  Since any $V \in \mathcal{T}_Y$ can be written as a union of basis elements $$V = \bigcup_{\alpha \in J}B_{\alpha}$$ Then $$f^{-1}(V) = \bigcup_{\alpha \in J} f^{-1}(B_{\alpha})$$ so that $f^{-1}(V) \in \mathcal{T}_X$ if each $f^{-1}(B_{\alpha}) \in \mathcal{T}_X$
  
  \item If the \textbf{topology}, $\mathcal{T}_Y$ on $Y$, \textbf{is given by a subbasis $\mathcal{S}$}, to prove the continuity of $f$ it will suffice to show that the inverse image of each \textit{subbasis element} is open. \\ \\ The arbitrary basis element $B$ for $Y$ can be written as a finite intersection $S_1 \cap .. \cap S_n$ of subbasis elements, it follows from the equation $f^{-1}(B) = f^{-1}(S_1) \cap .. \cap f^{-1}(S_n)$, so that the inverse image of every basis element is open.
\end{enumerate}

\bigskip

\subsection{Homeomorphisms}

\begin{definition}[\textbf{Homeomorphism}]


Let $(X, \mathcal{T}_X)$ and $(Y, \mathcal{T}_Y)$ be topological spaces, let $f : X \to Y$ be a \textit{bijection}. If both $f$ and the inverse function $f^{-1} : Y \to X$ are continuous, then $f$ is called a \textbf{homeomorphism}
\end{definition}

\begin{remark}[*Important]
Another way to define a \textbf{homeomorhpism} is to say that it is a bijective correspondence $f : X \to Y$, such that $f(U) \in \mathcal{T}_Y$ if and only if $U \in \mathcal{T}_X$.
\end{remark}

\begin{definition}[\textbf{Topological Imbedding}]
Suppose $f: X \to Y$ is an injective continuous map, $(X, \mathcal{T}_X)$ and $(Y, \mathcal{T}_Y)$ are topological spaces. Let $Z = f(X)$ be a subspace of  $(Y, \mathcal{T}_Y)$; then the function $f' : X \to Z$ obtained by restricting the range of $f$ to $Z$ is bijective. If $f'$ happens to be a homeomorphism of $X$ with $Z$, we say that the map $f: X \to Y$ is a \textbf{topological imbedding} or simply an \textbf{imbedding} of $X$ in $Y$.
\end{definition}

\newpage

\section{Interesting examples}

A collection of some interesting examples

\hrulefill


(1) \textit{Find a function $f: \mathbb{R} \to \mathbb{R}$ that is contiuous at precisely one point}. \\

The function
$$f(x)=\begin{cases}
x\text{ if }x\in\mathbb{Q}\\
0\text{ if }x\not\in\mathbb{Q}
\end{cases}$$

is continuous at $x=0$ but nowhere else.



\hrulefill


\newpage

\section{Appendix}

This reference appendix gives a list of some theorems that frequently pop up, but are out of the bounds of this text.

\begin{theorem}
If $f : A \to B$, and $A_0 \subset A$ and $B_0 \subset B$, then $$A_0 \subset f^{-1}(f(A_0)) \ \ \  \text{and} \ \ \ f(f^{-1}(B_0)) \subset B_0$$
\end{theorem}

\end{document}



